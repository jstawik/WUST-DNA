\documentclass[a4paper, 12pt]{article}
\usepackage[a4paper,left=2.5cm,right=2.5cm,top=2cm,bottom=2cm,marginpar=2.5cm]{geometry}
\usepackage{graphicx}
\usepackage[english]{babel}
\usepackage[T1]{fontenc}
\usepackage[utf8]{inputenc}
\usepackage{times}

\usepackage{graphicx}
\usepackage[dvipsnames]{xcolor}

\usepackage{amsmath}
\usepackage{amsfonts}
\usepackage{amssymb}
\usepackage{amsthm}
\usepackage{mathtools}

\usepackage{enumitem}
\renewcommand\labelitemii{--}
\renewcommand\labelitemi{$\circ$}

\usepackage{bm}

%% Asymptotic notations
\newcommand{\BigO}[1]{\mathrm{O}\left(#1\right)}
\newcommand{\BigTheta}[1]{\mathrm{\Theta}\left(#1\right)}

\usepackage[pdfauthor={Kubuś Puchatek},pdftitle={e2z1},%
pdfstartview={FitH}, pdftex, colorlinks=false, %plainpages=false,%pdfpagelabels=false, 
bookmarks={true}]{hyperref}
\usepackage[all]{hypcap}

\begin{document}

\section*{MSc Thesis}
What topics should be discussed / described in the thesis (these are merely some suggestions!).
\begin{enumerate}[label*=\arabic*.]
	\item introduction -- preliminaries, description of the studied problem and literature survey
		\begin{enumerate}[label*=\arabic*.]
			\item preliminaries -- notations used, some mathematical / theoretical definitions, main theorems etc.
			\item a short paragraph about wireless sensor networks (WSNs) / algorithmics of WSNs.
			\item distributed data aggregation (incl. large datasets / data streams) / data aggregation in WSNs
					\begin{itemize}
						\item formulation of the problem
						\item the idea of probabilistic counting and application of probabilistic counters to aggregation in large datasets
						\item description of the algorithm \textsc{HistMean} from \cite{JCi:2018:Histograms} and the approach
							  to averaging time-series data from \cite{Nugroho:2020:AveragingTS}
						\item brief mention of some other important algorithms for distributed data aggregation known from literature
					\end{itemize}
			\item simulations of distributed computing systems -- what are the key issues, some examples of existing solutions / distributed computing platform etc. 
		\end{enumerate}
	
	\item description of the implemented simulator (rather concise)
		\begin{enumerate}[label*=\arabic*.]
			\item motivations (\emph{``Why I've designed and implemented a custom solution instead of using one of the existing simulators?''}) and project objectives
			\item technical aspects of the implementation (short description of used technologies and design patterns), architecture, design objectives etc. 
			\item implemented features, main use cases, what it offers and how it works, possible directions for further extension / improvements
			\item comparison with existing solutions (if there will be enough time)
		\end{enumerate}
	
	\item averaging time-series data -- applications of the implemented simulator to the study of selected problems in distributed aggregation of large datasets in WSNs
		  by the example of building distributed data sketches and averaging of time-series data in WSNs
		  \begin{enumerate}[label*=\arabic*.]
		  	\item motivations and description of considered scenarios.
		  	\item description of considered variants of analyzed algorithms (some modifications of the approach from \cite{Nugroho:2020:AveragingTS}).
		  	\item quantities of interest -- which parameters / properties of considered algorithms will be analyzed and why? 
		  		\begin{itemize}
		  			\item estimator's precision
		  			\item communication complexity (message complexity) for selected families of graphs -- e.g. line graphs $L_n$, grid graphs $G_{k \times k}$,
		  				  random geometric graphs $\mathcal{R}_{n,r}$, 
		  			$d$-regular graphs, ``grid-of-cliques'', etc. $\rightarrow$ choose up to $2$--$3$ examples
		  		\end{itemize} 
		  \end{enumerate} 
	  
	 \item simulation results -- experimental evaluation of estimator's precision and message complexity
		 \begin{enumerate}[label*=\arabic*.]
		 	\item presentation (tables, graphs, charts, ...) and discussion on the obtained results of simulations for different scenarios
		 		\begin{itemize}
		 			\item different variants of the algorithm
		 			\item different input data (various distributions of the data, different network sizes $n$)
		 			\item parameter 1: precision of the obtained estimates of the average
		 			\item parameter 2: message complexity for certain network topologies
		 		\end{itemize}
	 		\item analysis of the results of performed experimental research, drawing some conclusions and stating hypotheses
		 \end{enumerate} 
	 
	 \item theoretical analysis of some selected properties of the considered protocols for averaging and building distributed data sketches 
	 		\begin{itemize}
	 			\item based on the obtained simulation results we should choose some quantity, e.g. estimator's precision, state some hypotheses and
	 					try to prove some theoretical results in the assumed formal model
	 			\item it suffices to show one -- two sample results
	 			\item e.g. something like ``In the basic variant $A$ of the averaging algorithms, the estimator's probabilistic error grows as $t \to \infty$''.
	 				  and ``In the strategy implemented in the variant $B$, the estimator's probabilistic error depends on the parameter $R$ and for constant $R$ tends
	 				  to $0$ in expectation as the numbers of counters $L$ grows.''
	 			\item what will be analyzed -- we'll take a look at the experimental results and decide, what seems to be achievable in 2-3 weeks (we don't have a lot of time)
	 		\end{itemize}
 		
 	\item Summary
 		\begin{itemize}
 			\item summary of the obtained experimental and theoretical results
 			\item conclusions
 			\item open problems and future work
 		\end{itemize}
	 
\end{enumerate}

\section*{Considered scenarios}
The network will be modeled by a simple, connected and undirected graph $G=(V,E)$ with $|V| = n$ vertices representing the stations and $|E| = m$ edges corresponding to
bidirectional links between nodes. Two stations $u, v \in V$ can communicate directly if and only if $\{u,v\} \in E$...

Assumptions on the considered theoretical model: 
\begin{itemize}
	\item \textbf{We'll discuss the details during the next call}
	\item each station observes a stream of data $(s^{(v)}_t)_{t \geq 0}$ arriving in consecutive time instants $t = 0, 1, \dotsc$-- like in \cite{Nugroho:2020:AveragingTS}
	\item our goal is to calculate the estimations $(avg_t)_{t\geq 0}$ in a fully distributed manner, where
		\[
			avg_t \stackrel{\text{df}}{=} \frac {\sum_{v \in V} s^{(v)}_t}{n}
		\]
	\item we will consider certain modifications of the method proposed in \cite{Nugroho:2020:AveragingTS}, which extends the approach based on approximate histograms
		introduced in \cite{JCi:2018:Histograms}.
	\item communication model -- like in \cite{JCi:2018:Histograms} (different model than in \cite{Nugroho:2020:AveragingTS}), i.e. broadcast-based communication instead of gossip-based
		(\textbf{this is only my suggestion -- you can choose what you prefer)}
	\item measures of precision -- I suggest implementing and calculating both measures of precision, i.e. $\text{err}(\text{wm}(\vec{C}_L,\text{wm}\vec{H})$ introduced in \cite{JCi:2018:Histograms}
		and $\eta$ from \cite{Nugroho:2020:AveragingTS}. We'll see what we'll get. In the thesis we can e.g. briefly mention, that the precision can be defined in different ways, 
		depending on the specific applications, considered scenarios etc.
	\item considered strategies (some modification of the approach of counting ``leavers'' and ``joiners'' for each interval):
		\begin{itemize}
			\item variant $0$ -- for each $t$ we independently run the \textsc{HistMean} algorithm with fixed $L$ and $K$
			\item variant $1$ -- the strategy from \cite{Nugroho:2020:AveragingTS} -- probabilistic counters for ``leavers'' $C^{leave}_{L,i}$ and ``joiners'' $C^{join}_{L,i}$
				for each interval $I_i$, $i \in \{1, \dotsc, K\}$  (the number of leavers and joiners will grow as
				$t \to \infty$ $\Rightarrow$ I suppose that the error related to probabilistic counters will also grow and the precision will be lost for large $t$,
				but this is only a conjecture and this may not be the case -- we should investigate this)\\
				Q: In \cite{Nugroho:2020:AveragingTS} there was a flag $r_{flag}$, which affects the exchange of messages -- should we embed this mechanism in the
				broadcast-based communication or not? We'll discuss it during the next call.
			\item variant $2$ -- ``reset'' the counters every $R^{\text{th}}$ round (e.g. $joiners \gets \max\{joiners - leavers,0\}$, $leavers \gets 0$)
			\item variant $3$ -- every $R^{\text{th}}$ round ``reset'' the counters by running the \textsc{HistMean} algorithm from scratch
		\end{itemize}
	\item in each case we will calculate the estimator's precision (as a function of the time $t$) for different network sizes $n$ (e.g. $n = 10,\,100,\,1000,\,5000$)
	\item distribution of the input data
		\begin{itemize}
			\item in the basic scenarios we assume that the data are independent (at least there is no spatial dependence)
			\item uniform distribution over some interval $[m, M]$ (e.g. over the unit interval)
			\item data for which the stations ``will change intervals'' rarely
			\item data for which the changes of intervals will be very frequent
		\end{itemize}
	\item for some scenarios and network topologies calculate the message complexity (the number of exchanged messages)
\end{itemize}

\section*{Future work}
\begin{itemize}
	\item data are generated with different intensities, some stations update their values more frequently than the others
	\item data streams are not time-homogeneous -- the distribution of the data $(s^{(v)}_t)_{t \geq 0}$ for a given station $v$ may vary in time
	\item spatial and temporal dependencies of data
	\item communication errors -- broken links, nodes failures
\end{itemize}

\bibliographystyle{alpha}
\bibliography{references}

\end{document}
